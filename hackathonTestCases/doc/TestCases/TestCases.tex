%%% Local Variables: 
%%% mode: latexTeX-master: t
%%% End: 
\documentclass{beamer}
\usepackage{graphicx}
\usepackage{natbib}
%\usepackage[latin1]{inputenc}
%\usepackage[french]{babel}
%\usepackage[T1]{fontenc}
%\usepackage{verbatim}
\usepackage[normalem]{ulem}
\usepackage{amsmath}
%\usepackage{beamerthemeMadrid}
%\usecolortheme{albatross}
%\usepackage{beamerthemeBoadilla} %pas mal du tout
%\usetheme{JuanLesPins}
\usetheme{Madrid}
%\usetheme{Bergen}
\DeclareMathAlphabet{\mathsl}{OT1}{cmss}{m}{sl}
\usepackage{color}
\usepackage{multicol}
%\usepackage{algorithms/algorithm}
%\usepackage{algorithms/algorithmic}
\usepackage{epsfig}
\usepackage{natbib}
\usepackage{graphicx}              % image
%\usepackage{here}                  % positionement facile des figures
%\usepackage[latin1]{inputenc}      % pour les caracteres accentués
%\usepackage[utf8x]{inputenc}
%\usepackage[francais]{babel}
%\usepackage[cyr]{aeguill}  %package permettant la cesure des mots français
%avec utilisation de font T1 avec bon rendu en pdf
%\usepackage[T1]{fontenc}
%\usepackage{lmodern}
\usepackage{fancyhdr}              %gestion entete /pied de pages
\usepackage{amsmath}
\usepackage{amsfonts}
%include pdf page
\usepackage{pdfpages}
\usepackage{fancybox}
\usepackage{pifont}
%\usepackage{multirow}

\definecolor{brique}{rgb}{0.5,0.2,0.4} 
\definecolor{highlight}{rgb}{1.,0.4,0.}

\newcommand{\RR}{\hbox{\cal I\hspace{-2pt}R}}
\newcommand{\ket}{\right\rangle}       
\newcommand{\bra}{\left\langle}

%\setbeamerfont{frametitle}{series=\bfseries,size=\large,fg=white}
\setbeamerfont{frametitle}{series=\bfseries,size=\large}
\setbeamercolor{structure}{bg=white, fg=brique}

%-------------------------------------------------------------------------------------

\title[Hackathon KAUST GPU 2020]{Hackathon KAUST GPU 2020}
%\subtitle{}
\author[Marcin, Suha \& Vincent]{Marcin Rogowski, Suha Kayum \& Vincent Etienne}
\institute[]{KAUST \& Saudi Aramco}

\date[Nov 2020] {\scriptsize{November 2020 }}

%-------------------------------------------------------------------------------------

\bibliographystyle{apalike}


\AtBeginSection[ ]
{
\begin{frame}<beamer>
\frametitle{Content}
\tableofcontents[currentsection]
\end{frame}
}

%\AtBeginSubsection[ ]
%{
%\begin{frame}<beamer>
%\frametitle{Content}
%\tableofcontents[currentsection,currentsubsection]
%\end{frame}
%}

%\titlegraphic{\vspace{-0.75cm} \center
%\pgfimage[height=2.cm]{logo/logo_complet.pdf}}

\begin{document}
\scriptsize

\maketitle 

\clearpage

\frame{
\frametitle{Content}
\tableofcontents }


%*************************************************************************************

\section{hpcscan}

%*************************************************************************************

%=====================================================================================
\subsection{Overview}
%=====================================================================================

%-------------------------------------------------------------------------------------
\frame{
  \frametitle{Overview}

  hpcscan is a C++ code for benchmarking HPC kernels (mainly for solving PDEs with FDM)

  \begin{itemize}
  \item Simple code struture based on individual test cases
  \item Easy to add new test cases
  \item Main class is Grid: multi-dimension (1, 2 \& 3D) Cartesian grid
  \item Hybrid MPI/OpenMP parallelism
  \item All configuration parameters on command line
  \item Support single and double precision computation
  \item Compilation with standard Makefile
  \item No external librairies
  \item Follows C++ Google style code
  \end{itemize}
}

%-------------------------------------------------------------------------------------
\frame{
  \frametitle{Overview}

  {\bf hpcscan embeds several test cases}
  \vspace{0.5cm}

  Current version 1.0
  \begin{itemize}
  \item General operations on grids
  \item Memory operations
  \item MPI communication
  \item FD computation 
  \item Basic wave propagator
  \end{itemize}
  
  Possible additions for future versions
  \begin{itemize}
  \item Operations on matrices full and sparse
  \item FFT
  \item IO
  \item Compression
  \end{itemize}
}

%=====================================================================================
\subsection{Compilation and validation}
%=====================================================================================

%-------------------------------------------------------------------------------------
\frame{
  \frametitle{Compilation and validation}

  Compiling hpcscan
  
  go to \texttt{./build} and \texttt{make} (by default compilation with single precision float)

  To compile with double precision float, \texttt{make precision=double}

  \vspace{0.5cm}
  Validating hpcscan

  go to \texttt{./script} and \texttt{sh runValidationTests.sh}

   {\tiny
  \begin{table}
    \caption{\scriptsize \texttt{runValidationTests.sh} \footnote{\scriptsize \textcolor{blue}{Updated Nov 25, 2020}} }
    \label{table_runAllTestCases}
    \begin{tabular}{@{}cccc}
      Machine & Compiler & Single prec. & Double prec. \\
      \hline
      Mars & g++ 9.3.0          & 764 PASS / 0 FAIL / 0 ERR / 20 WARN & 764 PASS / 0 FAIL / 0 ERR / 20 WARN \\
      Shaheen & icpc 19.0.5.281 & 764 PASS / 0 FAIL / 0 ERR / 20 WARN & 764 PASS / 0 FAIL / 0 ERR / 20 WARN \\
    \end{tabular}
  \end{table}
   }

   Numbers can differ due to availability of features depending on the platforms
}

%*************************************************************************************

\section{Test platforms}

%*************************************************************************************

%=====================================================================================
\subsection{Shaheen II (KAUST)}
%=====================================================================================

%-------------------------------------------------------------------------------------
\frame{
  \frametitle{Test platform - Shaheen II (KAUST)}

  {\bf Machine Shaheen II / Cray XC40}
  \begin{itemize}
  \item Computing nodes Intel Haswell 2.3 Ghz dual socket (16 cores / socket)
  \item RAM 128 GB with Peak memory BW 136.5 GB/s
  \item Peak performance Single Prec. 2.36 TFLOP/s / Double Prec. 1.18 TFLOP/s
  \item Interconnect Cray Aries with Dragonfly topology
    \begin{itemize}
    \item \scriptsize 60 GB/s optical links between groups
    \item \scriptsize 8.5 GB/s copper links between chassis
    \item \scriptsize 3.5 GB/s backplane within a chassis
    \item \scriptsize 5 GB/s PCIe from node to Aries router
    \end{itemize}
  \end{itemize}

  \begin{figure}
    \begin{center}
      \includegraphics[width=0.6 \textwidth]{./Images/Shaheen-II.png}
    \end{center}
  \end{figure}
}

%*************************************************************************************

\section{Test Case Grid}

%*************************************************************************************

%-------------------------------------------------------------------------------------
\frame{
  \frametitle{Test Case Grid - Description}

  \begin{itemize}
  \item Fill grid (W = coef)
  \item Max. err. grid W
  \item L1 err. grid W
  \item Get min. grid W
  \item Get max. grid W
  \item Update pressure (used in propagator)
  \item Small Grid size 500 MB (500 x 500 x 500 points)
  \item Medium Grid size 4 GB (1000 x 1000 x 1000 points)
  \end{itemize}
  
}

%-------------------------------------------------------------------------------------
\frame{
  \frametitle{Test Case Grid - Results}

  \begin{itemize}
  \item {\bf Machine: shaheen}
  \item 1 node / 32 threads
  \item Baseline kernel
  \end{itemize}
  
  {\scriptsize
  \begin{table}
    \caption{\scriptsize Bandwidth GB/s \footnote{\scriptsize \textcolor{blue}{Updated Nov 26, 2020}} }
    \label{table_memory}
    \begin{tabular}{@{}ccccccc}
      Grid   & Fill & Max. err. & L1 err. & Get max. & Get min. & Update Pres. \\
      \hline
      Small  & 58   & 72        & 122     & 125      & 125      & 119 \\
      Medium & 54   & 91        & 124     & 127      & 127      & 120 \\
    \end{tabular}
  \end{table}
  }

  {\scriptsize
  \begin{table}
    \caption{\scriptsize Bandwidth GPoints/s}
    \label{table_memory}
    \begin{tabular}{@{}ccccccc}
      Grid   & Fill & Max. err. & L1 err. & Get max. & Get min. & Update Pres. \\
      \hline
      Small  & 14.4  & 9.0      & 15.2    & 31.3     & 31.2    & 6.0 \\
      Medium & 13.4  & 11.4     & 15.5    & 31.8     & 31.8    & 6.0 \\
    \end{tabular}
  \end{table}
  }

  Reproduce results with \texttt{./script/testCase\_Grid/runSmallGridShaheen.sh}
  and \texttt{./script/testCase\_Grid/runMediumGridShaheen.sh}
  
  Elapsed time 5 and 7 sec.

}

%-------------------------------------------------------------------------------------
\frame{
  \frametitle{Test Case Grid - Summary}

  {\bf Machine: Shaheen}
  \begin{itemize}
  \item L1 Err., Get Min \& Max: 125 GB/s close to peak BW (92 \% Peak Mem. BW)
  \item {Low perf for Fill: 54-58 GB/s (40-43 \% Peak Mem. BW)}
  \item Max Err. 72-91 GB/s (53-67 \% Peak Mem. BW)
    \item Pressure update 6 GPoint/s (120 GB/s, 88 \% Peak Mem. BW)
  \end{itemize}
}

%*************************************************************************************

\section{Test Case Comm}

%*************************************************************************************

%-------------------------------------------------------------------------------------
\frame{
  \frametitle{Test Case Comm - Description}

  {\bf Measure MPI communication bandwidth}
  
  \vspace{0.5cm}
  MPI point to point communication
  \begin{itemize}
  \item Send with MPI\_Send from proc X to proc 0 (Half-duplex BW)
  \item Send and receive with MPI\_Sendrecv between proc X and proc 0 (Full-duplex BW)
  \end{itemize}

  MPI collective communication
  \begin{itemize}
  \item Exhange of halos used in FD kernel with MPI\_Sendrecv
  \item Grid size 1000 x 1000 x 1000
  \item Domain decomposition with N1 x N2 x N3 subdomains
  \end{itemize}
  
}


%-------------------------------------------------------------------------------------
\frame{
  \frametitle{Test Case Comm - Results}

  \begin{itemize}
  \item {\bf Machine: Shaheen}
  \item 8 MPI processes (1 per computing node)
  \item Baseline kernel
  \end{itemize}
  
  {\tiny
  \begin{table}
    \caption{\scriptsize Bandwidth GB/s \footnote{\scriptsize \textcolor{blue}{Updated Sep 19, 2020}} }
    \label{table_comm}
    \begin{tabular}{@{}ccccccc}
      MPI\#1 & MPI\#2 & Send & Sendrecv & Halo exch. & Comm. size & Subdomains \\
      \hline
      0 & 1 & 8.5 & 15.3 & -    & 47 MB  & - \\
      0 & 2 & 8.3 & 15.3 & -    & 47 MB  & - \\
      0 & 3 & 8.6 & 15.3 & -    & 47 MB  & - \\
      0 & 4 & 8.5 & 15.3 & -    & 47 MB  & - \\
      0 & 5 & 8.2 & 15.3 & -    & 47 MB  & - \\
      0 & 6 & 8.5 & 15.3 & -    & 47 MB  & - \\
      0 & 7 & 8.6 & 15.3 & -    & 47 MB  & - \\
      All & All & -    & -   & 5.0  & 128 MB & 1 4 2 \\
      All & All & -    & -   & 5.1  & 128 MB & 1 2 4 \\
      All & All & -    & -   & 2.0  & 96 MB  & 2 2 2 \\
    \end{tabular}
  \end{table}
  }

  Reproduce results with \texttt{./script/testCase\_Comm/runTestShaheen.sh}

  Elapsed time 9 seconds
}

%*************************************************************************************

\section{Test Case FD\_D2}

%*************************************************************************************

%-------------------------------------------------------------------------------------
\frame{
  \frametitle{Test Case FD\_D2 - Description}

  \begin{itemize}
  \item Computation of second order derivatives with finite-differnce stencil
  \item Directionnal derivatives
    \begin{itemize}
    \item {\scriptsize Axis 1 $W = \partial^2_{x1} (U)$}
    \item {\scriptsize Axis 2 $W = \partial^2_{x2} (U)$}
    \item {\scriptsize Axis 3 $W = \partial^2_{x3} (U)$}
    \end{itemize}
  \item Laplacian
    \begin{itemize}
    \item {\scriptsize For 2D grids $W = \Delta (U) =\partial^2_{x1} (U) + \partial^2_{x2} (U)$}
    \item {\scriptsize For 3D grids $W = \Delta (U) =\partial^2_{x1} (U) + \partial^2_{x2} (U) + \partial^2_{x3} (U)$}
    \end{itemize}
  \item Stencil order 2, 4, 8, 12 \& 16
  \item Grid size
    \begin{itemize}
    \item {\scriptsize Small 500 x 500 x 500}
    \item {\scriptsize Medium 1000 x 1000 x 1000}
    \end{itemize}
  \end{itemize}
  
}

%-------------------------------------------------------------------------------------
\frame{
  \frametitle{Test Case FD\_D2 - Results}

  \begin{itemize}
  \item {\bf machine Shaheen} /  1 node with 32 threads / Baseline kernel \footnote{\scriptsize \textcolor{blue}{Updated Sep 26, 2020}}
  \item \texttt{./script/testCase\_FD\_D2/runSmallGridShaheen.sh} \& \texttt{runMediumGridShaheen.sh}
  \end{itemize}

  \begin{figure}
    \begin{center}
      \includegraphics[width=0.85 \textwidth]{./Figures/FD_D2_ONE_PROC_SHAHEEN.jpg}
    \end{center}
  \end{figure}

}

%-------------------------------------------------------------------------------------
\frame{
  \frametitle{Test Case FD\_D2 - Results}

  \begin{itemize}
  \item {\bf machine Shaheen} /  1 node with 32 threads / Cache blocking kernel \footnote{\scriptsize \textcolor{blue}{Updated Sep 26, 2020}}
  \item \texttt{./script/testCase\_FD\_D2/runSmallGridShaheen.sh} \& \texttt{runMediumGridShaheen.sh}
  \end{itemize}

  \begin{figure}
    \begin{center}
      \includegraphics[width=0.85 \textwidth]{./Figures/FD_D2_ONE_PROC_SHAHEEN_CACHEBLK.jpg}
    \end{center}
  \end{figure}
  
}

%-------------------------------------------------------------------------------------
\frame{
  \frametitle{Test Case FD\_D2 - Results}

  \begin{itemize}
  \item {\bf machine Shaheen}
  \item 1 to 8 nodes with 32 threads/node
  \item Baseline kernel \footnote{\scriptsize \textcolor{blue}{Updated Sep 26, 2020}}
  \item Strong scalabity: Grid 1000 x 1000 x 1000 (4 GB)
  \item Weak scalabity: Grids from 4 GB (1 proc) to 32 GB (8 proc)
  \item 3D Laplacian O8
  \end{itemize}

  \begin{figure}
    \begin{center}
      \includegraphics[width=1.0 \textwidth]{./Figures/FD_D2_SCALABILITY_SHAHEEN.jpg}
    \end{center}
  \end{figure}

}

%-------------------------------------------------------------------------------------
\frame{
  \frametitle{Test Case FD\_D2 - Summary}
  {\bf machine Shaheen}
  \begin{itemize}
  \item Large benefit of cache blocking
  \item Significant effect of grid dimnsion and index (very bad performance for n3 without cache blocking)
  \item Min BW 50 GFLOP/s ($\partial^2_{x3}$ O2) = 2 \% peak BW [apparent Mem. BW 150 GB/s]
  \item Max BW 370 GFLOP/s ($\Delta$ O8) = 16 \% peak BW [apparent Mem. BW 900 GB/s]
  \item Apparent Mem. BW 150-900 GB/s (110-660 \% Peak Mem. BW) = shows data in-cache effect
  \item Typical stencils of interest for geophysical applications
    \begin{itemize}
    \item {\scriptsize $\Delta$ O4  BW = 8-10 GPoint/s}
    \item {\scriptsize $\Delta$ O8  BW = 7-9 GPoint/s}
    \item {\scriptsize $\Delta$ O12 BW = 3-5 GPoint/s}
    \end{itemize}
  \item Parallel efficiency with 8 nodes 55 to 86 \% (depends on workload on Shaheen)
  \end{itemize}
}

%*************************************************************************************

\section{Test Case PropaAc2}

%*************************************************************************************

%-------------------------------------------------------------------------------------
\frame{
  \frametitle{Test Case PropaAc2 - Results}
  
  \begin{itemize}
  \item {\bf machine Mars} / preliminary results \footnote{\scriptsize \textcolor{blue}{Updated Nov 5, 2020}}
  \item Eigen mode - 1D model
    \item FD: Black O2, Blue O4, Pink O8, Red O12 / Square=Baseline 
    \item \texttt{./paramAnalysis/propaAccuracy/runMars.sh} 
  \end{itemize}
  
  \begin{figure}
    \begin{center}
      \includegraphics[width=0.6 \textwidth]{./Figures/runMars_AccPerf.jpg}
    \end{center}
  \end{figure}

}

%*************************************************************************************

\section{Conclusions and next steps}

%*************************************************************************************

%-------------------------------------------------------------------------------------
\frame{
  \frametitle{Conclusions and next steps}

  TO DO
}

%*************************************************************************************

\section{Acknowledgements}

%*************************************************************************************

%-------------------------------------------------------------------------------------
\frame{
  \frametitle{Acknowledgements}

  \begin{itemize}
  \item KAUST ECRC and KSL for access and support on Shaheen II \& Ibex
  \end{itemize}
}

%-------------------------------------------------------------------------------------
%{\tiny{
%\def\newblock{\hskip .11em plus .33em minus .07em}
%\bibliographystyle{apalike}
%\bibliography{./biblioseiscope}
%}}

%\addtocounter{framenumber}{-4}

\end{document}

